% ****** Start of file apssamp.tex ******
%
%   This file is part of the APS files in the REVTeX 4.2 distribution.
%   Version 4.2a of REVTeX, December 2014
%
%   Copyright (c) 2014 The American Physical Society.
%
%   See the REVTeX 4 README file for restrictions and more information.
%
% TeX'ing this file requires that you have AMS-LaTeX 2.0 installed
% as well as the rest of the prerequisites for REVTeX 4.2
%
% See the REVTeX 4 README file
% It also requires running BibTeX. The commands are as follows:
%
%  1)  latex apssamp.tex
%  2)  bibtex apssamp
%  3)  latex apssamp.tex
%  4)  latex apssamp.tex
%
\documentclass[%
 reprint,
%superscriptaddress,
%groupedaddress,
%unsortedaddress,
%runinaddress,
%frontmatterverbose, 
%preprint,
%preprintnumbers,
%nofootinbib,
%nobibnotes,
%bibnotes,
 amsmath,amssymb,
 aps,
%pra,
%prb,
%rmp,
%prstab,
%prstper,
%floatfix,
]{revtex4-2}
\usepackage{lipsum}
\usepackage{caption}
\usepackage{subcaption}
\usepackage{siunitx}
\usepackage{gensymb}
\sisetup{separate-uncertainty=true}
\usepackage{graphicx}% Include figure files
\usepackage{dcolumn}% Align table columns on decimal point
\usepackage{bm}% bold math
\usepackage{hyperref}% add hypertext capabilities
\usepackage{xcolor}
\hypersetup{
	colorlinks,
	linkcolor={blue},
	citecolor={red},
	urlcolor={purple}
}
%\usepackage[mathlines]{lineno}% Enable numbering of text and display math
%\linenumbers\relax % Commence numbering lines

%\usepackage[showframe,%Uncomment any one of the following lines to test 
%%scale=0.7, marginratio={1:1, 2:3}, ignoreall,% default settings
%%text={7in,10in},centering,
%%margin=1.5in,
%%total={6.5in,8.75in}, top=1.2in, left=0.9in, includefoot,
%%height=10in,a5paper,hmargin={3cm,0.8in},
%]{geometry}

\begin{document}

\preprint{APS/123-QED}

\title{Experiments in Fiber Optics}% Force line breaks with \\
%\thanks{A footnote to the article title}%

\author{Maitrey Sharma}
\email{maitrey.sharma@niser.ac.in}
\thanks{\\Roll No.: 1911093}
% \altaffiliation[Also at ]{Physics Department, XYZ University.}%Lines break automatically or can be forced with \\
%\author{Second Author}%
% \email{Second.Author@institution.edu}
\affiliation{%
 School of Physical Sciences, National Institute of Science Education and Research, HBNI, Jatni-752050, India.\\
 %This line break forced with \textbackslash\textbackslash
}%
%\collaboration{MUSO Collaboration}%\noaffiliation

%\author{Charlie Author}
% \homepage{http://www.Second.institution.edu/~Charlie.Author}
%\affiliation{
% Second institution and/or address\\
% This line break forced% with \\
%}%
%\affiliation{
% Third institution, the second for Charlie Author
%}%
%\author{Delta Author}
%\affiliation{%
% Authors' institution and/or address\\
% This line break forced with \textbackslash\textbackslash
%}%

%\collaboration{CLEO Collaboration}%\noaffiliation

\date{\today}% It is always \today, today,
             %  but any date may be explicitly specified

\begin{abstract}
In this experiment, we have explored a simple method to measure non-linear properties of different
optical materials - Single Beam Zscan. The experiments were performed with a $ \mathrm{TEM}_{00} $ Gaussian laser
with a wavelength of 532 nm. We have analyzed the data using Python and calculated the non-Linear
Refractive Index  and the non-Linear absorption  coefficient for diffrent samples. We have
also examined certain improvements to our setup that could give better results.
%\begin{description}
%\item[Usage]
%econdary publications and information retrieval purposes.
%\item[Structure]
%You may use the \texttt{description} environment to structure your abstract;
%use the optional argument of the \verb+\item+ command to give the category of each item. 
%\end{description}
\end{abstract}

%\keywords{Suggested keywords}%Use showkeys class option if keyword
                              %display desired
\maketitle

%\tableofcontents

\section{Introduction}


In the early 1840s Paris, Daniel Colladon and Jacques Babinet first demonstrated the guiding of light by refraction and by the 19th century, a team of doctors from Vienna were able to guide light through bent glass rods to illuminate body cavities. Over the next century practical applications followed and in 1953, Dutch scientist Bram van Heel first demonstrated image transmission through bundles of optical fibers with a transparent cladding. 

Today, optical fibers form the backbone of our communications systems.
\section{Objectives}
	There are several major objectives that will be achieved as part of this experiment. They are:
	\begin{enumerate}
		\item To study the basics of non-
		linear optical properties by
		using Z-scan technique.
		\item Taking measurements of transmitted power for open and
		closed aperture by translating
		the material in the $ z $-direction.
		\item By fitting these data with the
		appropriate formulas, we will
		find the medium's nonlinear
		absorption coefficient and non-
		linear refractive index for different samples.
		\item To compare the difference between non-linear refractive index of a sample in two forms: as a \textit{thin film coating} and as a \textit{solution}. 
	\end{enumerate}


	
	

\section{Theory}
	
		
		
\section{The Experiments}
		
\section{Discussions}
	\begin{enumerate}
		\item While taking observations, we noticed that there were
		some places near the focus where we found some sudden fluctuations in transmittance. To remove the dip, we
		tried reducing the power. The magnitude of the fluctuations was somewhat minimized but they could not be removed completely. 
		\item During data analysis, we employed Python's SciPy package and used its subroutine to remove the fluctuations (based on the Savitzky–Golay filter) for a better fit to the our theoretical model. 
		\item In the case of the organic sample in thin film form, there were multiple dips after the focus resulting in sub-par data. In case of solution of the same sample, the data obtained was not presentable due
		to too much dominance of the multiple interference
		fluctuations. 
		\item This may be due to high absorption coefficient
		for the sample so that the refractive index changes
		very rapidly due to significant thermal variation along
		\item We performed the experiment for the same sample and verified that the refractive index and absorption coefficient were within the margin of error. The discrepancies could be explained by the relative instability of the cuvette when performing the experiment with solution.
	\end{enumerate}
	
	
\section{Conclusions}
	\begin{enumerate}
		\item We can use Z-scan experimental configuration to obtain nonlinear refractive
		index and nonlinear absorption coefficients of any standard samples. 
		\item The
		sign and magnitude of nonlinear refractive index of the samples can measured. 
		\item Using the equations of normalized transmittance and fitting
		them with the data points and will report the refractive index and nonlinear
		absorption coefficients.
		\item The property of non-linear sample to change refractive index on changing intensity can be used to make a optical transistor
		type of thing.
	\end{enumerate}

\section{Precautions and sources of error}
	

	
	


\appendix

% The \nocite command causes all entries in a bibliography to be printed out
% whether or not they are actually referenced in the text. This is appropriate
% for the sample file to show the different styles of references, but authors
% most likely will not want to use it.
\nocite{*}

\bibliography{apssamp}% Produces the bibliography via BibTeX.

\end{document}
%
% ****** End of file apssamp.tex ******
